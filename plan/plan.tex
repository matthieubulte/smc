%%% Local Variables:
%%% mode: latex
%%% TeX-master: t
%%% End:

\documentclass{article}
\usepackage{cite}
\usepackage{amssymb}
\usepackage{mathtools}
\usepackage[utf8]{inputenc}

\usepackage[ruled,lined]{algorithm2e}

\begin{document}

\date{}
\title{Sequential Monte Carlo for time-dependent Bayesian Inverse Problems}
\author{Matthieu Bulté}
\maketitle

This document is a detailed plan of the Bachelor thesis I intend to write under the supervision of Prof. Ullmann and J. Latz covering Sequential Monte Carlo methods for static Bayesian inverse problems.

The goal of this Bachelor thesis will be to give the mathematical reader a self-contained guide for infering static parameters of time-dependent models from a set of observations. To do this, I will present the Bayesian inverse problem associated to the time-dependent inference task, as well as theoretical properties of this formulation. I will follow with a presentation and analysis of Sequential Monte Carlo methods as a numerical approximation of the solution of the Bayesian inverse problem. I will then conclude with a practical application on a real-world inference problem in order to illustrate theoretical results and assess performances of the numerical method.

The plan of the Bachelor thesis will be the following:

\begin{itemize}
\item{Introduction}
  \begin{itemize}
  \item Inverse problem and ill-posedness, refering to Hadamard's definition \cite{hadamard}
  \item Pendulum inverse problem 
  \item Litterature review
  \end{itemize}
  
\item{Bayesian Inverse Problem}
  \begin{itemize}
  \item Filtering problem for static problems, similar to \cite{stuart_2010} and \cite{chopin_2002}
  \item Time-dependent Bayesian updating 
  \item Characterization of the solution and its well-posedness
  \end{itemize}

\item{Sequential Monte Carlo}
  \begin{itemize}
  \item Monte Carlo to approximate a solution
  \item Importance Sampling and Sequential Importance Sampling
  \item Sequential Monte Carlo, presenting the general method defined in \cite{del_moral_2006}
  \item Kernel choice (focus on MCMC kernels)
  \item Proof of convergence, probably an adaptation of \cite{beskos_2015}
  \end{itemize}

\item{Application}
  \begin{itemize}
  \item Presentation and bayesian formulation of the Pendulum inverse problem
  \item Proof that the problem is well-posed
  \item Presentation and analysis of the SMC numerical solution
  \end{itemize}
\end{itemize}

\bibliographystyle{plain}
\bibliography{references}

\end{document}
