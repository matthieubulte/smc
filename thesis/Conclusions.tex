\section{Conclusion}

We have presented inverse problems and filtering problems for cases where more data becomes available over time. We have shown how the Bayesian framework provides a well-posed solution to the inverse problem and how Bayesian filtering gives a natural and sound formalism to incorporate new data to the existing knowledge about the unknown parameters. We have then shown how common numerical estimation algorithms fail at efficiently approximating the sequence of filtering posterior distributions and have constructed the SMC algorithm to provide an estimator that is not only asymptotically correct but also solves many practical difficulties encountered when running the algorithm with finitely many particles available for approximation.

However, many areas could be explored to extend this work. Firstly, we have used a very simple model of the pendulum. We could define a new model to take into account air friction and treat the friction coefficient of the pendulum's mass as an additional parameter of the model. Since the pendulum was let go by a human operator in the experiments, there is also uncertainty in the initial angle of the pendulum which could be taken into account in the model. These additional modelling choices could not only improve the quality of our estimate but could also test the performance of the SMC algorithm on higher-dimensional models with latent variables. Secondly, we did not elaborate upon the choice of Markov kernels for the MCMC, SIS and SMC algorithms. While Metropolis-Hastings kernels are a very common choice, constructing such Markov chains has been a very active research area from which it could be possible to use more recent results. Also, the use of Markov kernels was only justified by empirical observations and we have not given any theoretical proof of the impact of these kernels on the convergence properties of the algorithms.

%%% Local Variables:
%%% mode: latex
%%% TeX-master: "Thesis"
%%% End:
