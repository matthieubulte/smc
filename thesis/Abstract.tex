\thispagestyle{plain}

\section*{Abstract}

The goal of this paper is to present a sound solution for solving inverse problems. We focus on the scenario where data becomes available over time, called filtering problem. Formulated in the Bayesian framework, we prove that the inverse and filtering problems are well-posed under mild assumptions. We study the Sequential Monte Carlo algorithm to approximate the solution to Bayesian filtering problems and compare it to several related approximation algorithms. The use of the Sequential Monte Carlo algorithm is validated through proofs of convergence and numerical experiments. We also illustrate the theoretical and numerical concepts presented in the paper by a concrete filtering problem, in which we attempt to estimate the parameter of a dynamical system from measurements of the system.

\section*{Zusammenfassung}

In dieser Arbeit besch�ftigen wir uns damit, eine korrekte und effiziente L�sung zu inversen Problemen zu finden. Wir konzentrieren uns auf Filtrierungsprobleme, wo Messpunkte im Laufe der Zeit zur Verf�gung gestellt werden. Mithilfe des Bayesschen Formalismus beweisen wir dass, unter schwachen Vorausetzungen, inverse Probleme und Filtrierungsprobleme gut gestellt sind. Wir studieren den Sequential Monte Carlo Algorithmus zur Approzimierung der L�sung des Bayesschen Filtrierungsproblems und vergleichen es zu �hnlichen Approximierungsalgorithmen. Wir best�tigen die Wahl des Sequential Monte Carlo Algorithmus durch Konvergenzbeweise und numerische Experimente. Die theoretischen und numerischen Konzepte werden auch durch ein konkretes Filtrierungsproblem dargestellt, indem wir versuchen aus Messpunkten, Parameter eines dynamischen Systems zu sch�tzen. 

%%% Local Variables:
%%% mode: latex
%%% TeX-master: "Thesis"
%%% End:
