\section{Introduction}
\pagenumbering{arabic}
\setcounter{page}{1}
\thispagestyle{empty}

Study of complex system if often done through mathematical modelling, allowing to simulate, analyse and predict their behaviour. These mathematical require input parameters, about which only limited or no information is known. Finding these parameters from measurements of the system is called the \textit{inverse problem}. Since measurements are often noisy or sparse, and the mathematical models complex and expensive to evaluate, developing sound and efficient mathematical frameworks to treat the inverse problem is a rich and difficult task.

Kaipio and Somersalo \cite{kaipio2006statistical} presents a rich introdoction to solving inverse problems in science and engineering. A rough classification of these methods would be to consider on the one side 

Furthermore, Kaipio and Somersalo \cite{kaipio2006statistical}, focusing on inverse problems in scientific computing, presents the frequentist viewpoint of the subject, and  also includes a large number of case studies of science and engineering inverse problems.


Inspired by \cite{bayes-tutorial}, this thesis is designed to be a rigorous but approachable guide for practitioners interested in solving time-dependent inverse problems related to dynamical systems modelled with ordinary differential equations (ODEs) often found in fields such as biology, robotics and many others.


this thesis will be structured around  solving a time-dependent inverse problem in order to illustrate and validate the theoretical concepts and numerical algorithms presented. The inverse problem considered is of estimating the \textit{gravitational acceleration} in the mathematical model describing 

Gelman et al. \cite{gelman} provide a thourough and more general presentation of the field of Bayesian statistics. Furthermore, Kaipio and Somersalo \cite{kaipio2006statistical}, focusing on inverse problems in scientific computing, presents the frequentist viewpoint of the subject, and  also includes a large number of case studies of science and engineering inverse problems.

The theory of well-posedness invistigated in this thesis was first presented by Hadamard \cite{hadamard} in his work on characterizing solvability of partial differential equations. Stuart \cite{stuart_2010}, uses this definition for developing a theory validating the use of the Bayesian approach for inverse problems in differential equations, focussing on .

Finally, Del Moral et al. \cite{del_moral_2006} give a general presentation of the SMC algorithm. More work from Del Moral \cite{del2013mean, del2004feynman} provides convergence result of the algorithm, and Beskos et al. \cite{beskos2015sequential} provides a simplified version of these proof together with a case study of an inverse problem associated to elliptic partial differential equations. 

This thesis will start with a presentation of \textit{Bayesian data analysis}, a methodology with probabilistic modeling as its core, expressing the solution of the inverse problem as a probability distribution of the parameters, called the \textit{posterior distribution}. We will show how to express inverse problems in this framework using the example of the estimation of the gravitational acceleration. Furthemore, this section will characterize the set of problems having a \textit{well-posed} solution, using Hadamard's \cite{hadamard} definition, and will proove well-posedness of the pendulum inverse problem. For most practical applications the posterior distribution does not have a closed form solution, creating the need to consider numerical approximations of the real posterior. In Section 3, a presentation of a particle based approximation algorithm, the \textit{Sequential Monte Carlo} (SMC) algorithm, will be presented. Its construction will show to provide a natural and efficient way of estimating the real posterior distribution. This section will also provide a proof of convergence of the approximation to the exact solution, and illustrate these results by approximating the solution to the pendulum's inverse problem.


\note{literature review could be improved by saying how they are related to what im doing}

\note{maybe more focus on filtering?}

%%% Local Variables:
%%% mode: latex
%%% TeX-master: "Thesis"
%%% End:
