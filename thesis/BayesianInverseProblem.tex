\section{Bayesian Inverse Problem}
\subsection{Overview}

This section will introduce 




A more mathematical formulation of the inverse problem is the following: given some \textit{observation} $y \in Y$, what is the parameter $x \in X$ solving the equation

\begin{equation}
  y = \mathcal{G}(x)
\end{equation}

where $\mathcal{G}$ is the \textit{}, mapping values from the \textit{parameter space} $X$ to the \textit{data space} $Y$, with $X,Y$ vector spaces of finite dimension.





thesis is a simple pendulum, an idealized model for a pendulum in which the mass of the pendulum and its friction are ignored, described by the following second-order and non-linear differential equation;

\begin{equation} \label{pendulum-diff-eq}
  \frac{\text{d}^2\theta}{\text{d}t^2} = -\frac{g}{l}\sin(\theta),
\end{equation}

where $\theta$ is the angle of the pendulum to its resting point, $l$ is the length of the pendulum and $g$ is the acceleration due to gravity, called \textit{gravitational acceleration}. The inverse problem will be to estimate the gravitational acceleration from an data of an experiment in which the pendulum is let go with no initial velocity from an angle $\theta_0$, and where to measurements are the times $(t_1, \ldots, t_n)$ at which the pendulum reaches a zero angle.

%%% Local Variables:
%%% mode: latex
%%% TeX-master: "Thesis"
%%% End:
